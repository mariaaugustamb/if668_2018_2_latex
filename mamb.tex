
\documentclass[10pt,a4paper]{article}

\usepackage[brazil]{babel}
\usepackage[utf8x]{inputenc}
\usepackage[T1]{fontenc}

\usepackage[a4paper,top=3cm,bottom=2cm,left=3cm,right=3cm,marginparwidth=1.75cm]{geometry}

\usepackage{amsmath}
\usepackage{graphicx}
\usepackage[colorinlistoftodos]{todonotes}
\usepackage[colorlinks=true, allcolors=blue]{hyperref}

\title{IF686 - Paradigmas de Linguagens Computacionais}
\author{Maria Augusta Mota Borba}

\begin{document}
\maketitle

\begin{figure}[h]
\centering
\includegraphics[width=0.5\textwidth]{linguagens.png}
\caption{\label{fig:linguagens.png}CC0 Creative Commons}
\end{figure}


\section{Introdução}

A disciplina de Paradigmas de Linguagens Computacionais - IF686 é lecionada, para o curso de Ciência da Computação, pelo Prof. Márcio Lopes Cornélio. Durantes o curso, os alunos aprenderão sobre paradigmas alternativos ao imperativo, sendo capazes de apresentar uma melhor compreensão das construções utilizadas pelas diversas linguagens computacionais. Apesar de não ter foco em nenhuma linguagem específica, apresenta uma certa ênfase em Haskell e o paradigma funcional e Java e o paradigma concorrente. As avaliações são realizadas através de exercícios escolares, listas de exercícios e aulas práticas. Os livros-textos base da displina são \cite{JavaConcurrencyinPractice} e \cite{LearnYouAHaskellForGreatGood} e, além disse, o site da matéria é \cite{SiteCadeira}, lá é disponibilizado diversos materiais que facilitam o aprendizado.

\section{Relevância}

Apesar de não focar em uma linguagem computacional específica, através desse curso você poderá aprender melhor a como utilizar as diversas linguagens, além de desenvolver uma melhor habilidade para a resolução de problemas e uma maior facilidade para aprender novas linguagens. 


\section{Relação com outras disciplinas}
\begin{center}
\begin {tabular}{|c|c|} 
\hline
Código da disciplina & Relação\\
\hline\hline
LE 530 & Facilita o entendimento dos comandos da maioria das linguagens. \\ 
\hline
IF 710& Ajuda a relacionar componentes para produçao de um sistema mais complexo.\\
\hline
IF 708 & Tem sua introdução sobre o paradigma funcional vista nessa cadeira. \\
\hline
IF 712& Ajuda a escolher a linguagem adequada para seu projeto. \\
\hline
IF 734& Facilita a criaçaõ de DSLs.\\
\hline
IF 722& Auxilia o entendimento e a documentação da arquitetura do software.\\
\hline
IF 724& Possibilita um melhor entendimento do que pode/deve ser reutilizado.\\
\hline
\end{tabular}
\end{center}


\nocite{CInWIKI}
\bibliographystyle{plain}
\bibliography{references}

\end{document}